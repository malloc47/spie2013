\newcommand{\stdfigure}[3]{ \begin{figure}[htp] \centering \includegraphics[width=0.95\linewidth]{#1} \caption{#3} \label{fig:#2} \end{figure} }
\newcommand{\stdfigurevar}[4]{ \begin{figure}[htp] \centering \includegraphics[width=#4\linewidth]{#1} \caption{#3} \label{fig:#2} \end{figure} }
\newcommand{\stdfigurefull}[3]{ \begin{figure*}[htp] \centering \includegraphics[width=0.95\linewidth]{#1} \caption{#3} \label{fig:#2} \end{figure*} }
\newcommand{\stdfigurefullvar}[4]{ \begin{figure*}[htp] \centering \includegraphics[width=#4\linewidth]{#1} \caption{#3} \label{fig:#2} \end{figure*} }

\newcommand{\fig}[1]{Figure~\ref{fig:#1}}
\newcommand{\figf}[1]{Figure~\ref{fig:#1}}
\newcommand{\figsub}[2]{Figure~\ref{fig:#1}\,(#2)}
\newcommand{\figsubf}[2]{Figure~\ref{fig:#1}\,(#2)}
\newcommand{\figsubref}[2]{Figure~\ref{fig:#1}\,\subref{fig:#2}}
\newcommand{\figsubreff}[2]{Figure~\ref{fig:#1}\,\subref{fig:#2}}
% not figure how to remove () from subref
\newcommand{\figsubrefrange}[3]{Figure~\ref{fig:#1}\,\subref{fig:#2}-\subref{fig:#3}}
\newcommand{\figsubrefrangef}[3]{Figure~\ref{fig:#1}\,\subref{fig:#2}-\subref{fig:#3}}

\newcommand{\eq}[1]{Eq.~(\ref{eq:#1})}
\newcommand{\eqf}[1]{Equation~(\ref{eq:#1})}

\newcommand{\sect}[1]{Section~\ref{sec:#1}}
\newcommand{\sectf}[1]{Section~\ref{sec:#1}}

\newcommand{\tbl}[1]{Table~\ref{tbl:#1}}
\newcommand{\tblf}[1]{Table~\ref{tbl:#1}}

\newcommand{\dataset}[1]{Dataset~#1}
\newcommand{\datasetf}[1]{Dataset~#1}

\newcommand{\alg}[1]{Algorithm~\ref{alg:#1}}
\newcommand{\algf}[1]{Algorithm~\ref{alg:#1}}
% \newcommand{\algl}[2]{Algorithm~\ref{alg:#1} line #2}
% \newcommand{\alglf}[2]{Algorithm~\ref{alg:#1} line #2}
% \newcommand{\algls}[2]{Algorithm~\ref{alg:#1} lines #2}
% \newcommand{\alglsf}[2]{Algorithm~\ref{alg:#1} lines #2}
\newcommand{\lne}[1]{line~\ref{ln:#1}}
% \newcommand{\lns}[1]{lines #1}
% \newcommand{\lnf}[1]{Line #1}
% \newcommand{\lnsf}[1]{Lines #1}

\newcommand{\red}[1]{\textcolor{red}{#1}}
\newcommand{\yellow}[1]{\textcolor{red}{#1}}
\newcommand{\green}[1]{\textcolor{green}{#1}}

\newcommand{\rem}[1]{\pdfmarkupcomment[markup=StrikeOut,color=red,author=JW]{#1}{removed}}
% \newcommand{\add}[1]{\pdfmarkupcomment[markup=Highlight,author=JW,color=yellow,opacity=1.0]{#1}{added}}
\newcommand{\add}[1]{\yellow{#1}}
% \newcommand{\upd}[1]{\pdfmarkupcomment[markup=Highlight,author=JW,color=green,opacity=1.0]{#1}{to update}}
\newcommand{\upd}[1]{\green{#1}}
\newcommand{\com}[2]{\pdfmarkupcomment[markup=Highlight,author=JW,color=blue,opacity=1.0]{#1}{#2}}

\newcommand{\ncite}[1]{}
\newcommand{\ncaption}[1]{}

\newcommand{\prop}[1]{Property~(#1)}
\def\props{properties}
\def\fprops{Properties}

\def\data{unary}
\def\smooth{binary}
\def\dataf{Unary}
\def\smoothf{Binary}

\def\obj{substructure}

\def\pslice{U_k}
\def\nslice{U_{k+1}}

\def\eg{e.g.}
\def\etal{et al.}
\def\etc{etc.}
\def\ie{i.e.}

\pdfminorversion=4
